\documentclass[11pt,a4paper]{article}
\usepackage[utf8]{inputenc}
\usepackage[T1]{fontenc}
\usepackage[provide=*]{babel}
\usepackage[german]{babel}
\usepackage{geometry}
\geometry{margin=1in}
\usepackage{graphicx}
\usepackage{xcolor}
\usepackage{fancyhdr}
\usepackage{tcolorbox}
\usepackage{hyperref}
\hypersetup{
    colorlinks=true,
    linkcolor=blue,
    urlcolor=blue
}
\usepackage{enumitem}
\usepackage{booktabs}

% Custom colors
\definecolor{forestgreen}{rgb}{0.13, 0.55, 0.13}
\definecolor{lightblue}{rgb}{0.68, 0.85, 0.9}

% Fancy header
\pagestyle{fancy}
\fancyhf{}
\fancyhead[L]{\leftmark}
\fancyhead[R]{\thepage}

% Title page
\title{Finde Deinen Weg! \\ \large Systemische Coaching Retreats in der Natur}
\author{Anja Heinicke}
\date{2025}

\begin{document}

\maketitle
\thispagestyle{empty}
\newpage

\section{Finde Deinen Weg!}

\subsection{Systemische Coaching Retreats in der Natur}

\begin{tcolorbox}[colback=lightblue!20, colframe=forestgreen, title=Eine Einladung zur Veränderung]
\begin{quote}
\emph{So geht es nicht weiter! -- Aber wie?}
\end{quote}

In unseren Coaching Retreats in der Natur können Sie Ideen und Ansätze entwickeln, welche Wege Sie in Zukunft gehen möchten.

Die äußere Ruhe der Natur kann Ihnen helfen, auch Ihre innere Ruhe zu finden. Lassen Sie die friedliche Atmosphäre auf sich wirken und kommen Sie mit sich in tiefem, inneren Kontakt.

Die Schönheit der Natur kann Ihnen helfen, Ihre innere Schönheit zu erkennen und wertzuschätzen. Wie eine Perle, die versteckt ist und entdeckt werden muss.

Die Aufmerksamkeit auf die Einfachheit zu lenken, sie wertzuschätzen und tiefe Zufriedenheit darin zu finden, ist eine zentrale Ausrichtung dieses Coaching-Retreats.
\end{tcolorbox}

\begin{figure}[h]
\centering
\includegraphics[width=0.8\textwidth]{../assets/images/hero-image.jpg}
\caption{Naturlandschaft für Coaching Retreat}
\end{figure}

\newpage

\section{Das Angebot}

\begin{tcolorbox}[colback=lightblue!10, colframe=blue, title=Unsere Angebote im Überblick]
\subsection{Klein und Persönlich}
Wir halten die Gruppe bewusst klein (max. 4 Personen), um einen geschützten Raum zu schaffen und intensive Auseinandersetzung zu ermöglichen.

\subsection{Natur als Therapeutin}
Jeden Tag besuchen wir unterschiedliche Orte in der Region. Die Schönheit und Vollkommenheit der Natur dient als Quelle der Inspiration und Kraftschöpfung.

\subsection{Raum für Entfaltung}
Das Retreat ist bewusst mit Freiräumen gestaltet, damit sich individuelle Entwicklungen entfalten können. Bewegung in der Natur bringt Gedanken und Ideen in Bewegung.
\end{tcolorbox}

\begin{figure}[h]
\centering
\includegraphics[width=0.8\textwidth]{../assets/images/nature-image.jpg}
\caption{Natur als Therapeutin}
\end{figure}

\newpage

\section{Coaching Themen}

Im systemischen Coaching können wir folgende Themen bearbeiten:

\begin{table}[h]
\centering
\begin{tabular}{@{}ll@{}}
\toprule
Thema & Beschreibung \\
\midrule
Burnout \& Prävention & Erschöpfung erkennen, verstehen und vorbeugen \\
Self-Care & Selbstfürsorge lernen und im Alltag verankern \\
Grenzen ziehen & Persönliche Grenzen setzen und kommunizieren \\
Selbstannahme & Sich selbst annehmen, wie Sie sind \\
Entscheidungsfindung & Klarheit gewinnen für wichtige Entscheidungen \\
Selbstwertgefühl & Den eigenen Wert erkennen und stärken \\
Selbstoptimierung hinterfragen & Vom Optimierungsdruck zur Selbstakzeptanz \\
Eigenverantwortung & Selbstwirksamkeit und Handlungsfähigkeit stärken \\
Coping Strategien & Bewältigungsstrategien für herausfordernde Zeiten \\
\bottomrule
\end{tabular}
\caption{Coaching Themen im Überblick}
\end{table}

\newpage

\section{Ablauf des 5-Tage Retreats}

\subsection{Tag 1: Ankommen}
Ankunft, Ankommen und erste Orientierung in der neuen Umgebung.

\subsection{Tage 2–4: Kernprogramm}
\begin{itemize}
\item 8:00 Uhr – Optionale gemeinsame Meditation
\item 8:30 Uhr – Gemeinsames Frühstück
\item 9:30 Uhr – Coaching-Sitzung
\item 13:00 Uhr – Pranzo (Mittagessen)
\item Ab 14:00 Uhr – Individuelle Tagesgestaltung, Ausflüge an sehenswerte Orte, gemeinsame Abendessen
\end{itemize}

\subsection{Tag 5: Abschluss}
Reflexion, Integration der Erfahrungen und Verabschiedung.

\subsection{Zusätzliche Angebote}
\begin{itemize}
\item Ausflüge an sehenswerte Orte in der Region mit anschließendem gemeinsamen Abendessen
\item Gemeinsame Meditationen zur Stressbewältigung und Achtsamkeit
\item Raum für individuelle Aktivitäten und Selbstreflexion
\end{itemize}

\newpage

\section{Über mich und meine Arbeitsweise}

\begin{tcolorbox}[colback=lightblue!15, colframe=green, title=Meine Philosophie]
\subsection{Systemische Therapeutin, Supervisorin und Coach}
Achtsamkeit, Akzeptanz und Wertschätzung bilden die Grundsäulen meiner Arbeit. Diese erlauben in einem geschützten Rahmen, sich für neue Erfahrungen zu öffnen und sich auf Veränderungsprozesse einzulassen.

Eine ressourcenorientierte Haltung ist grundlegend für meine Arbeit. Jede Person verfügt über Ressourcen, die eventuell verdeckt sind. Im Coaching können diese wieder entdeckt und zugänglich gemacht werden.

Im Coaching können Sie Ihre Aufmerksamkeit auf Ihr inneres Wissen legen, welches vielleicht von äußeren Erwartungshaltungen überlagert ist, und können so Ihren inneren Frieden wiederfinden.

\subsection{Die Frage nach dem "Statt dessen"}
Wünsche nach "Verschwinden" von Problemen, Schwierigkeiten oder Symptomen sind nachvollziehbar – doch oft tritt das Gegenteil ein und sie treten verstärkt auf.

Die systemische Frage nach dem "Statt dessen" nimmt daher großen Raum in unserem Prozess ein und ermöglicht neue Perspektiven.

\textbf{Ihr inneres Wissen ist wertvoll. Es lohnt sich, diesem zu vertrauen und es als Kompass für Ihre Entscheidungen zu nutzen.}
\end{tcolorbox}

\begin{figure}[h]
\centering
\includegraphics[width=0.6\textwidth]{../assets/images/about-image.jpg}
\caption{Anja Heinicke}
\end{figure}

\newpage

\section{Kontakt}

\begin{tcolorbox}[colback=lightblue!20, colframe=blue, title=Kontaktieren Sie mich]
\textbf{Anja Heinicke} \\
Systemische Therapeutin | Supervisorin | Coach \\
Spezialisiert auf systemisches Coaching in der Natur für persönliche Neuausrichtung und Ressourcenentdeckung.

\textbf{E-Mail:} \href{mailto:anja.heinicke.coaching@gmail.com}{anja.heinicke.coaching@gmail.com}

Ich freue mich auf Ihre Nachricht!
\end{tcolorbox}

\vspace{1cm}

\textit{Quelle: Kombiniert und verbessert aus vorhandenen PDF-Dokumenten}

\end{document}